% !TEX encoding = UTF-8 Unicode
% !TEX root =  Bachelorarbeit.tex

% Anpassung des Seitenlayouts ---------------------------------------------------
%   siehe Seitenstil.tex
% -----------------------------------------------------------------------------------------
\usepackage[
    automark, % Kapitelangaben in Kopfzeile automatisch erstellen
    headsepline, % Trennlinie unter Kopfzeile
    ilines % Trennlinie linksbündig ausrichten
]{scrlayer-scrpage}

\usepackage{scrhack}

% Anpassung an Landessprache -------------------------------------------------
\usepackage[ngerman]{babel}


% Umlaute ------------------------------------------------------------------------------
%   Umlaute/Sonderzeichen wie äüöß direkt im Quelltext verwenden (CodePage).
%   Erlaubt automatische Trennung von Worten mit Umlauten.
%   Für Umlaute siehe Hauptdokument Zeile 35
% -----------------------------------------------------------------------------------------
\usepackage{textcomp} % Euro-Zeichen etc.


% Schrift --------------------------------------------------------------------------------
\usepackage{lmodern} % bessere Fonts
\usepackage{relsize} % Schriftgröße relativ festlegen


% Bessere Unterstreichungen ---------------------------------------------
\usepackage[normalem]{ulem}


% Grafiken -----------------------------------------------------------------------------
% Einbinden von JPG-Grafiken ermöglichen
\usepackage[dvips,final]{graphicx}
% hier liegen die Bilder des Dokuments
\graphicspath{{Bilder/}}


% Befehle aus AMSTeX für mathematische Symbole z.B. \boldsymbol \mathbb
\usepackage{amsmath,amsfonts}


% Eurozeichen benutzen ------------------------------------------------------------
\usepackage{eurosym}


% für Index-Ausgabe mit \printindex -----------------------------------------------
\usepackage{makeidx}


% Einfache Definition der Zeilenabstände und Seitenränder etc. ------------
\usepackage{setspace}
\usepackage{geometry}


% Symbolverzeichnis ------------------------------------------------------------------
%
%   Symbolverzeichnisse bequem erstellen. Beruht auf MakeIndex:
%     makeindex.exe %Name%.nlo -s nomencl.ist -o %Name%.nls
%   erzeugt dann das Verzeichnis unter WIndows. Dieser Befehl kann z.B. im TeXnicCenter
%   als Postprozessor eingetragen werden, damit er nicht ständig manuell
%   ausgeführt werden muss.
%
%   Unter Unix bitte einfach das mitgelieferte Shellskript "makemyindex" nutzen.
%
%   Die Definitionen werden im Fließtext mit den Befehlen:
%               - \Fachbegriff
%               - \FachbegriffSpezial
%               - \FachbegriffSpezialB
%    erzeugt oder separat in die Datei "Glossar.tex" eingetragen.
% -------------------------------------------------------------------------------------------
\usepackage[intoc]{nomencl}
\let\abbrev\nomenclature
\renewcommand{\nomname}{Abkürzungsverzeichnis und Glossar}
\setlength{\nomlabelwidth}{.25\hsize}
\renewcommand{\nomlabel}[1]{#1 \dotfill}
\setlength{\nomitemsep}{-\parsep}


% zum Umfließen von Bildern ---------------------------------------------------------
\usepackage[vflt]{floatflt}
\usepackage{subfigure}

% zum Einbinden von Programmcode -----------------------------------------------
\usepackage{listings, mdframed}
\usepackage{xcolor} 
\definecolor{colKeys}{rgb}{0.8,0,0.5}
\definecolor{colIdentifier}{rgb}{0.6,0,0.3}
\definecolor{colComments}{rgb}{0,0.5,0}
\definecolor{colString}{rgb}{0,0,1}

\lstset{
    float=htbp,
    basicstyle=\ttfamily\color{black}\small\smaller,
    identifierstyle=,%\color{colIdentifier},
    keywordstyle=\color{colKeys}\bfseries,
    stringstyle=\color{colString},
    commentstyle=\color{colComments},
    columns=flexible,
    tabsize=4,
    frame=lines,
    extendedchars=true,
    showspaces=false,
    showstringspaces=false,
    numbers=left,
    numberstyle=\tiny,
    breaklines=true,
    breakautoindent=true,
    escapeinside={(*@}{@*)},
    literate={Ö}{{\"O}}1 {Ä}{{\"A}}1 {Ü}{{\"U}}1 {ß}{{\ss}}2 {ü}{{\"u}}1 {ä}{{\"a}}1 {ö}{{\"o}}1 {µ}{\textmu}1
 }

% --------------------------------------------------------------------------------------------
% 
% Eigene Definitionen für Quelltext-Stile
%

% Define Smalltalk
\lstdefinelanguage{Smalltalk}{
    morekeywords={self,super,true,false,nil,thisContext},
    morestring=[d]',
    morecomment=[s]{"}{"},
    alsoletter={\#:},
    upquote=true,
    showstringspaces=false,
    literate=
    %     {.}{{\bfseries .}}1
    %     {;}{{\bfseries ;}}1
    %     {[}{{\bfseries [}}1
    %     {]}{{\bfseries ]}}1
    %     {:}{{\bfseries :}}1
    % %    {|}{{\ttfamily\textbar{}}}1
    %     {\#}{{\ttfamily \#}}1
    %     {\$}{{\ttfamily \$}}1
    %     {:=}{{$\mathrel{\mathop:}=$\ }}2%$\shortleftarrow$\ }}2
    %     {^}{{\raisebox{0.5ex}{$\wedge$}}}1%\raisebox{1.5ex}{\scalebox{1}[-1]{$\lightning$}}\,}}2%\textuparrow\ }}2
    {>>}{{\,}>>{\,}}3
    {>>>}{>>>}3,
    tabsize=4
}[keywords,comments,strings]
% URL verlinken, lange URLs umbrechen etc. -------------------------------------
\usepackage{url}


% natbib einbinden -----------------------------------------------------------------------
\usepackage[square,numbers,sort&compress]{natbib}


% PDF-Optionen -------------------------------------------------------------------------
\usepackage[
    bookmarks,
    bookmarksopen=true,
    colorlinks=true,
% diese Farbdefinitionen zeichnen Links im PDF farblich aus
    linkcolor=red, % einfache interne Verknüpfungen
    anchorcolor=black,% Ankertext
    citecolor=blue, % Verweise auf Literaturverzeichniseinträge im Text
    filecolor=magenta, % Verknüpfungen, die lokale Dateien öffnen
    menucolor=red, % Acrobat-Menüpunkte
    urlcolor=cyan, 
%
% diese Farbdefinitionen sollten für den Druck verwendet werden (alles schwarz):
%
    %linkcolor=black, % einfache interne Verknüpfungen
    %anchorcolor=black, % Ankertext
    %citecolor=black, % Verweise auf Literaturverzeichniseinträge im Text
    %filecolor=black, % Verknüpfungen, die lokale Dateien öffnen
    %menucolor=black, % Acrobat-Menüpunkte
    %urlcolor=black, 
%
    backref,
    plainpages=false, % zur korrekten Erstellung der Bookmarks
    pdfpagelabels, % zur korrekten Erstellung der Bookmarks
    hypertexnames=false, % zur korrekten Erstellung der Bookmarks
    linktocpage % Seitenzahlen anstatt Text im Inhaltsverzeichnis verlinken
]{hyperref}
% Befehle, die Umlaute ausgeben, führen zu Fehlern, wenn sie hyperref als Optionen übergeben werden
\hypersetup{
    pdftitle={\titel \untertitel},
    pdfauthor={\autor},
    pdfcreator={\
    },
    pdfsubject={\titel \untertitel},
    pdfkeywords={\keywords},
}

% Paket zum sauberen Einbauen von externen PDF-Dateien -----------------
\usepackage[final]{pdfpages}


% fortlaufendes Durchnummerieren der Fußnoten, Bilder und Tabellen -------------------------------
\usepackage{chngcntr}


% schönere Tabellen --------------------------------------------------------------------
\usepackage{tabularx}
\usepackage{multirow}


% für lange Tabellen ---------------------------------------------------------------------
\usepackage{longtable}
\usepackage{ltxtable}
\usepackage{filecontents}
\usepackage{array}
\usepackage{ragged2e}
\usepackage{lscape}


% Rotation von Elementen -------------------------------------------------------
\usepackage{rotating}


% Formatierung von Listen ändern --------------------------------------------------
\usepackage{paralist}
\usepackage{enumitem}


% bei der Definition eigener Befehle benötigt -------------------------------------
\usepackage{ifthen}
\usepackage{forloop}


% definiert u.a. die Befehle \todo und \listoftodos --------------------------------
\usepackage{todonotes}


% sorgt dafür, dass Leerzeichen hinter parameterlosen Makros nicht als Makroendezeichen interpretiert werden
\usepackage{xspace}

% ermöglicht Zeichnungen
\usepackage{tikz}

% emögliche Verbatim-Umgebungen
\usepackage{verbatim}
\usetikzlibrary{shapes,backgrounds}

% mache "Figure"/"Abb." fett und ohne ":" am Ende
\usepackage{caption}

% notwenig, um Figures wirklich an der aktuellen Stelle zu platzieren
\usepackage{placeins}

% für \toprule, \midrule, \bottomrule in Tabellen
\usepackage{booktabs}

% ermöglicht mit \cref bessere Referenzen
\usepackage[noabbrev,capitalize]{cleveref}