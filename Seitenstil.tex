% !TEX encoding = UTF-8 Unicode
% !TEX root =  Bachelorarbeit.tex

% Zeilenabstand 1,5 Zeilen ---------------------------------------------------------------------
%\onehalfspacing{}

% Seitenränder ------------------------------------------------------------------------------------
\setlength{\topskip}{\ht\strutbox} % behebt Warnung von geometry
\geometry{paper=a4paper,left=20mm,right=20mm,top=20mm}


% Kopf- und Fußzeilen --------------------------------------------------------------------------
\pagestyle{scrheadings}

% Kopf- und Fußzeile auch auf Kapitelanfangsseiten
\renewcommand*{\chapterpagestyle}{scrheadings} 

% Schriftform der Kopfzeile
\renewcommand{\headfont}{\normalfont}

% Abstand vor und hinter Überschriften verringern
\RedeclareSectionCommand[
  afterindent=false,
  beforeskip=0pt,
  afterskip=2\baselineskip]{chapter}
\RedeclareSectionCommand[
  afterindent=false,
  beforeskip=\baselineskip,
  afterskip=.5\baselineskip]{section}
\RedeclareSectionCommand[
  afterindent=false,
  beforeskip=.75\baselineskip,
  afterskip=.5\baselineskip]{subsection}
\RedeclareSectionCommand[
  afterindent=false,
  beforeskip=.5\baselineskip,
  afterskip=.25\baselineskip]{subsubsection}
\RedeclareSectionCommand[
  runin=true,
  beforeskip=.5\baselineskip,
  afterskip=1em]{paragraph}
\RedeclareSectionCommand[
  runin=true,
  beforeskip=.5\baselineskip,
  afterskip=1em]{subparagraph} 


% Kopfzeile über den Text hinaus verbreitern
\setlength{\headheight}{21mm} % Höhe der Kopfzeile

% Kopfzeile für einseitiges Dokument:
\ihead{%\small{\textsc{\titel}  \\
  %\untertitel
  %\\[2ex]
  \textit{\headmark}%}
}
\chead{}
% Für Kopfzeilen mit nur einem Logo
% \ohead{\includegraphics[scale=0.06]{\logo}}
% Für Kopfzeilen mit zwei Logos
\ohead{\includegraphics[scale=0.06]{\logo} \includegraphics[scale=0.06]{\logozwei}}


%% Kopfzeile für Zweiseitiges Dokument:
%\ihead{\textit{\leftmark}}
%\chead{}
%\ohead{}
%\lehead{\includegraphics[scale=0.25]{\logo}}
%\rohead{\includegraphics[scale=0.25]{\logo}}
%\setheadsepline[text]{0.4pt} % Trennlinie unter Kopfzeile

% Fußzeile
%\ifoot{\copyright\ \autor}
\cfoot{}
\ofoot{\pagemark \\[4ex]}
\setlength{\footheight}{15mm} % Höhe der Fußzeile


% sonstige typographische Einstellungen ---------------------------------------------------
% erzeugt ein wenig mehr Platz hinter einem Punkt
\frenchspacing{}

% Schusterjungen und Hurenkinder vermeiden
\clubpenalty = 10000
\widowpenalty = 10000 
\displaywidowpenalty = 10000

% Quellcode-Ausgabe formatieren
\lstset{numbers=left, numberstyle=\tiny, numbersep=5pt, breaklines=true}
\lstset{emph={square}, emphstyle=\color{red}, emph={[2]root,base}, emphstyle={[2]\color{blue}}}

% Fußnoten, Figures und Tabllen fortlaufend durchnummerieren
\counterwithout{footnote}{chapter}
\counterwithout{figure}{chapter}
\counterwithout{table}{chapter}

% Eine horizontale Linie über den Captions erzeugen, "Figure" wird bold und ohne ":" vor der eigentlichen caption
\DeclareCaptionFormat{captionformat}{\hrulefill\\#1#2#3}
\captionsetup[figure]{labelfont=bf, labelsep=space, format=captionformat} 
\captionsetup[table]{labelfont=bf, labelsep=space, format=captionformat} 

% Vermeiden, dass Fußnoten auf mehrere Seiten verteilt werden
\interfootnotelinepenalty = 9999